\chapter*{Úvod}
\phantomsection
\addcontentsline{toc}{chapter}{Úvod}

Měřiče impedance, často označované jako \acs{LCR} metry, představují klíčové vybavení jak pro 
vývojové laboratoře, tak pro výrobní linky. Jejich primárním účelem je precizní 
měření impedance a odvozených veličin, jako jsou ztrátový činitel, \acs{ESR} či 
činitel kvality.

Tato semestrální práce se zabývá návrhem a ověřením konceptu 
měřiče impedance. Hlavní motivací pro vznik tohoto zařízení 
byla absence cenově dostupného řešení na trhu, které by zároveň nabízelo 
dostatečnou přesnost a možnost snadné integrace s počítačem.

Na základě rešerše komerčně dostupných přístrojů a analýzy požadavků na měření 
byly stanoveny klíčové parametry navrhovaného zařízení. 
Cílem je vytvořit přístroj, který se svými vlastnostmi přiblíží profesionálním 
řešením, avšak s výrazně nižšími pořizovacími náklady.

Specifikace navrženého zařízení: 
\begin{itemize} 
    \item Frekvenční rozsah: 100 Hz -- 100 kHz 
    \item Rozsah měření indukčnosti: 100 nH -- 10 H 
    \item Rozsah měření kapacity: 100 pF -- 10 mF 
    \item Cílová přesnost: 1\% 
    \item Amplituda měřicího signálu: 0,1 -- 2 V 
    \item Stejnosměrná složka (Bias): 0 -- 2 V 
    \item Konektivita: Komunikace s PC a napájení přes USB 
\end{itemize}

Text práce je logicky členěn do několika částí. 
První kapitola shrnuje teoretická východiska týkající se impedance a přehled měřicích metod. 
Druhá kapitola se zaměřuje na implementaci analogového měřicího obvodu včetně zdroje signálu a samo-vyvažovacího můstku. 
Třetí část rozebírá problematiku digitalizace a zpracování signálu. 
Čtvrtá kapitola popisuje první prototyp zařízení, ověřené subsystémy a zjištěné nedostatky návrhu. 
V závěru jsou shrnuty dosažené výsledky a navrženy další kroky pro navazující bakalářskou práci.