\chapter*{Úvod}
\phantomsection
\addcontentsline{toc}{chapter}{Úvod}

V elektronické praxi se často setkáváme s požadavkem na měření impedance, 
od jednoduché identifikace hodnot součástek až po plnou charakterizaci kondenzátorů či cívek.

Měřiče impedance, často označované jako \acs{LCR} metry, představují klíčové vybavení
jak pro vývojové laboratoře, tak pro výrobní linky.
Komerční \acs{LCR} metry dosahují vysoké přesnosti a nabízejí široké 
spektrum funkcí, avšak jejich cena je často příliš vysoká pro běžné 
použití v~domácích laboratořích a~při vývoji prototypů. Tato 
skutečnost motivovala k~návrhu vlastního měřicího zařízení, které by 
při rozumných nákladech poskytovalo dostatečnou přesnost pro běžné 
aplikace.

Práce se nejprve věnuje teoretickým základům měření impedance, 
včetně popisu různých měřicích metod se zaměřením na metodu 
samo-vyvažovacího můstku. Dále je popsána implementace zdroje 
měřicího signálu a jsou rozebrány 
metody digitalizace naměřených signálů. V~praktické části je 
představen první prototyp zařízení, na kterém byly ověřeny navržené 
koncepty. Testování prototypu odhalilo několik nedostatků, které 
jsou v~práci analyzovány a na základě kterých jsou navrženy další 
kroky pro navazující bakalářskou práci.
