\chapter{Teoretická část}

\section{Impedance}

Impedance $Z$ je fyzikální veličina, která popisuje celkový odpor elektrického obvodu vůči střídavému proudu o dané frekvenci.
Vyjadřuje se komplexním číslem, kdy reálná část odpovídá odporu $R$ a imaginární část reprezentuje reaktanci $X$.

V některých případech je výhodné použít obrácenou hodnotu impedance $Y = \frac{1}{Z}$, zvanou admitance. \cite{KeysightImpedanceHandbook}

\section{Metody měření impedance}

\subsection{Můstková metoda}

Měření pomocí můstku je jednoduché a přesné, avšak velmi nepraktické.
Princip metody spočívá v tom, že pokud jsou impedance obou větví ($Z2$ a $Zx$) stejné,
pak musí být napětí mezi větvemi nulové. Schéma můstku je zobrazeno v obr. \ref{fig:wheatstone}. \cite{KeysightImpedanceHandbook}

\begin{figure}[h!]
    \centering
    \includesvg[width=0.55\textwidth]{obrazky/wheatstone.svg}
    \caption{Schéma měřícího můstku}
    \label{fig:wheatstone}
\end{figure}

Když se $Z2$ nastaví tak, aby mezi větvemi bylo naměřeno nulové napětí, pak je měřená
impedance $Zx = \frac{Z1}{Z3}\cdot Z2$

Problém této metody spočívá v tom, že je nutno znát nastavenou hodnotu $Z2$.
Protože nastavitelné kondenzátory a cívky nejsou praktické, používá se tato metoda
zejména pro měření malých změn v odporu např. tenzometru.

\subsection{Rezonanční metoda}


\subsection{Metoda měření proudu a napětí}
Princip této metody je výpočet hodnoty neznámé impedance z naměřených hodnot napětí a proudu podle ohmova zákona. \cite{KeysightImpedanceHandbook}
V praxi se proud měří pomocí měření napětí na bočníku.
$$|Z| = \frac{U_{1}}{I} = \frac{U_{1}}{U_{2}}\cdot R$$
Takovým měřením však získáme pouze modul komplexní impedance, z čehož nelze určit, 
zda je charakter impedance rezistivní nebo reaktivní.
Proto je nutné měřit také fázový posun $U_{1}$ a $U_{2}$, který je roven argumentu $\varphi$.
$$ Z = |Z| \angle \varphi$$

\subsection{Samo-vyvažovací můstek}
V praxi se metoda měření proudu a napětí často implementuje pomocí tzv. samo-vyvažovacího můstku,
což je operační zesilovač, který vytváří virtuální zem \vizobr{fig:autobalance_bridge}. \cite{KeysightImpedanceHandbook}

Výhoda takového uspořádání je, že jak měřená impedance tak bočník jsou připojeny k zemi, 
je tedy možné na nich jednoduše měřit napětí vzhledem k zemi.

\begin{figure}[h!]
    \centering
    \includegraphics[width=0.55\textwidth]{obrazky/i_v_conv.pdf}
    \caption{Schéma samo-vyvažovacího můstku}
    \label{fig:autobalance_bridge}
\end{figure}


\section{Měřiče impedance dostupné na trhu}
Na trhu je mnoho dostupných zařízení pro měření impedance. Pro ukázku byly vybrány tři,
každé ze samostatné cenové kategorie.

\subsection{Univerzální tester komponent}

Jako zástupce nejlevnější třídy byl vybrán tester LCR-T4 \vizobr{fig:lcr-t4}. Tento přístroj neměří přímo impedanci,
ale měří hodnoty odporu, kapacity a indukčnosti. Jako doplněk má také funkci rozpoznání tranzistorů.

Pro měření využívá metodu časové konstanty, kdy se měří čas nabití kondenzátoru přes známý odpor.

\begin{figure}[h!]
    \centering
    \includegraphics[width=0.6\textwidth]{obrazky/LCR-T4.jpg}
    \caption{Univerzální tester komponent LCR-T4 \cite{hadex_lcrt4}}
    \label{fig:lcr-t4}
\end{figure}

Přesnost měření není udávána, přístroj totiž slouží pouze pro hrubou identifikaci součástek.
Cena se pohybuje okolo 350 Kč\cite{hadex_lcrt4}.

\subsection{Ruční LCR měřiče}

Pokročilejší přístroje jsou ruční měřiče LCR ve formě podobné multimetrům. Pro příklad
byl vybrán měřič Keysight U1732C \vizobr{fig:keysight_lcr}.

Ruční měřiče jsou vhodné pro rychlou identifikaci cívek a kondenzátorů, nabízejí specifikovanou přesnost
a možnost nastavit parametry měření jako frekvenci měřícího signálu. \cite{Keysight_U1732C_Datasheet}


\begin{figure}[h!]
    \centering
    \includegraphics[width=0.35\textwidth]{obrazky/Keysight_U1732C.jpg}
    \caption{Ruční měřič LCR Keysight U1732C \cite{Keysight_U1732C_Datasheet}}
    \label{fig:keysight_lcr}
\end{figure}

\subsection{Stolní LCR měřiče}

Pro nejpřesnější měření je nutno použít špičkové stolní LCR měřiče. 
Jako zástupce byl vybrán přistroj LCX200 od firmy Rohde\&Schwarz \vizobr{fig:rhode_lcr}.

Tyto přístroje podporují čtyřbodovou metodu měření, která eliminuje parazitní indukčnost
měřících sond. Narozdíl od ručních měřičů se zde používají stíněné koaxiální sondy, které jsou méně náchylné k rušení.

Výrobce standardně dodává několik různých fixtur k připojení měřeného vzorku, typicky SMD součástky.

Většina stolních měřičů také podporuje připojení do sítě LAN a automatizaci měření. \cite{RS_LCX_Datasheet}

\begin{figure}[H]
    \centering
    \includegraphics[width=0.85\textwidth]{obrazky/Rhode_schwarz-LCX200.jpg}
    \caption{Stolní měřič Rohde\&Schwarz LCX200 \cite{RS_LCX_Datasheet}}
    \label{fig:rhode_lcr}
\end{figure}

V tabulce \ref{tab:comparison} jsou porovnány přístroje Keysight U1732C a Rohde \& Schwarz LCX200.
Stolní přístroj je výrazně přesnější a nabízí pokročilejší možnosti měření, je ale i násobně dražší.

\begin{table}[h!]
    \centering
    \caption{Porovnání LCR metrů \cite{RS_LCX_Datasheet} \cite{Keysight_U1732C_Datasheet}}
    \label{tab:comparison}
    \vspace{0.2cm}
    
    % Definice sloupců:
    % l = zarovnání doleva (pro první sloupec, aby byl kompaktní)
    % X = sloupec, který vyplní zbývající místo a zalamuje text
    % >{\raggedright\arraybackslash} = zajistí zarovnání vlevo uvnitř X sloupce (lépe vypadá než plné zarovnání do bloku u úzkých sloupců)
    \begin{tabularx}{\textwidth}{@{} l >{\raggedright\arraybackslash}X >{\raggedright\arraybackslash}X @{}}
    \toprule
    \textbf{Parametr}               & \textbf{Keysight U1732C}                  & \textbf{Rohde \& Schwarz LCX200}                  \\ \midrule
    \textbf{Konstrukce}             & Ruční                                     & Stolní                                            \\ \addlinespace
    \textbf{Cena}                   & cca $13\,000$ – $15\,000$ Kč              & cca $120\,000$ – $300\,000$ Kč+                   \\ \addlinespace
    \textbf{Frekvenční rozsah}      & Pevné body: $100$ Hz, $120$Hz, $1$ kHz, $10$ kHz & Spojitý: $4$ Hz až $10$ MHz                       \\ \addlinespace
    \textbf{Základní přesnost}      & $0,2\,\%$                                 & $0,05\,\%$                                        \\ \addlinespace
    \textbf{Amplituda test. signálu}& Pevná (typicky $0,74$ Vrms)               & Nastavitelná: $10$ mV až $10$ V                   \\ \addlinespace
    \textbf{DC Bias (Předpětí)}     & Není k dispozici                          & Interní $0$--$10$ V (Externí až $40$ V)           \\ \addlinespace
    \textbf{Rychlost měření}        & $\sim 1$--$2$ měření/s                    & až $250$ měření/s                   \\ \addlinespace
    \textbf{Konektivita}            & IR-to-USB                                 & USB, LAN, Digital I/O, GPIB                       \\ \bottomrule
    \end{tabularx}
    
    
\end{table}