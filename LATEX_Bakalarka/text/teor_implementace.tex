\chapter{Implementace samo-vyvažovacího můstku}

\section{Zdroj měřícího signálu}

Pro účely měření impedance je nezbytné vybudit měřenou komponentu adekvátním napětím a proudem.

Specifikace návrhových požadavků definují, že budicí zdroj musí generovat harmonické 
střídavé napětí s plně nastavitelnými parametry, zahrnujícími frekvenci, 
amplitudu a superponovanou stejnosměrnou složku (DC offset).


\subsection{Přímá digitální syntéza (\acs{DDS})}
V moderních měřičích impedance se téměř výhradně používá pro generování sinusoidního signálu
metoda DDS. Historicky nebyly digitální integrované obvody dostatečně výkonné, proto se využívaly
různé analogové oscilátory a harmonické tvarovače. 
Takové obvody byly však složité, dnes se již implementují jen zřídka.

Metoda DDS spočívá v generování sinusového průběhu \acs{D/A} převodníkem, do kterého jsou 
z paměti načítány předvypočítané hodnoty funkce $sin(t)$. Tyto hodnoty se nazývají \uv(vyhledávací tabulka),
nebo anglicky \uv(lookup table) (\acs{LUT}).
Frekvence výsledného signálu je závislá na prodlevě mezi načítáním jednotlivých hodnot. \cite{ArtOfElectronics}
Signál z převodníku je následně filtrován dolní propustí o adekvátní mezní frekvenci,
která odstraní vyšší harmonické složky vzniklé z diskrétních digitálních kroků.
Princip funkce DDS je ilustrován na obr. \ref{fig:DDS} 

\begin{figure}[h!]
    \centering
    \includegraphics[width=1\textwidth]{obrazky/DDS_chart.pdf} % Adjusted width for the second figure
    \caption{Ilustrace principu DDS}
    \label{fig:DDS} % Changed label for the second figure
\end{figure}


\subsection{Úprava amplitudy a stejnosměrné složky signálu}
Jednoduché integrované obvody DDS typicky nepodporují nastavení amplitudy nebo 
stejnosměrné složky signálu, tuto úpravu je tedy nutno provést externě.

Digitálního řízení amplitudy lze docílit několika způsoby, například:
\begin{itemize}
    \item Zapojit sinusový signál jako referenci \acs{D/A} převodníku, který pak bude fungovat jako digitálně ovládaný dělič. 
    \item Použít zesilovač s programovatelným zesílením (\acs{PGA}). Tyto zesilovače většinou mají málo úrovní zesílení.
    \item Použít operační zesilovač s digitálním potenciometrem ve zpětné vazbě. Takové zapojení je limitováno minimálním zesílením $A=1$.
    \item Operačním zesilovačem zesílit amplitudu na maximální hodnotu a následně ji snížit digitálním potenciometrem zapojeným jako odporový dělič.
    \item Nejdříve zeslabit signál digitálním potenciometrem -- děličem a poté jej zesílit operačním zesilovačem s fixním zesílením. Oproti předchozímu zapojení se tím ušetří jeden sledovač, protože výstup operačního zesilovače je nízkoimpedanční.
\end{itemize}

Z těchto metod se jako nejvýhodnější jeví poslední jmenovaná. Její schéma je zobrazeno na obr. \ref{fig:ampl}.

\begin{figure}[h!]
    \centering
    \includegraphics[width=0.7\textwidth]{obrazky/amplitude_set_sch.pdf} % Adjusted width for the second figure
    \caption{Schéma obvodu pro nastavení amplitudy signálu.}
    \label{fig:ampl} % Changed label for the second figure
\end{figure}
Fixní zesílení operačního zesilovače se řídí vztahem \ref{eq:noninv_gain}.
\begin{equation}
    A =\frac{V_{out}}{V_{in}} = 1 + \frac{R_2}{R_3}
    \label{eq:noninv_gain}
\end{equation}
Vstupní amplituda je dána $U_{in} = 0,6 \ \unit{V}$, požadovaná maximální amplituda signálu je $U_{out} =3\ V$.
Dosazením (rovnice \ref{eq:noninv_dosaz}) získáme poměr rezistorů $R_{2}$ a $R_{3}$.
\begin{equation}
    A =\frac{3}{0,6} = 5 = 1 + \frac{4}{1}
    \label{eq:noninv_dosaz}
\end{equation}
Prakticky můžeme použít např. $R_{3} = 10\ k\Omega$ ; $R_{2} = 4\times 10\ k\Omega$ v sérii.

Celková amplituda výstupního signálu $U_{\text{out}}$ vychází z rovnice \ref{eq:outputamp}.
Digitální hodnotu potenciometru pro konkrétní amplitudu $U_{out}$ vypočítáme ze vztahu \ref{eq:digival}.

\begin{equation}
    U_{\text{out}} = U_{\text{in}} \cdot \frac{X}{2^n} \cdot A
\label{eq:outputamp}
\end{equation}
\begin{equation}
        X = \frac{2^n\cdot U_{\text{out}}}{A\cdot U_{\text{in}}} 
    \label{eq:digival}
\end{equation}
\hspace{11em}\text{kde } $n$ \text{ je bitové rozlišení digitálního potenciometru}








