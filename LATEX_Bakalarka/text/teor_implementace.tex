\chapter{Implementace samo-vyvažovacího můstku}

\section{Zdroj měřícího signálu}

Pro účely měření impedance je nezbytné vybudit měřenou komponentu adekvátním napětím a proudem.

Specifikace návrhových požadavků definují, že budicí zdroj musí generovat harmonické 
střídavé napětí s plně nastavitelnými parametry, zahrnujícími frekvenci, 
amplitudu a superponovanou stejnosměrnou složku (DC offset).


\subsection{Přímá digitální syntéza (\acs{DDS})}
V moderních měřičích impedance se téměř výhradně používá pro generování sinusoidního signálu
metoda DDS. Historicky nebyly digitální integrované obvody dostatečně výkonné, proto se využívaly
různé analogové oscilátory a harmonické tvarovače. 
Takové obvody byly však složité, dnes se již implementují jen zřídka.

Metoda DDS spočívá v generování sinusového průběhu \acs{D/A} převodníkem, do kterého jsou 
z paměti načítány předvypočítané hodnoty funkce $sin(t)$. Tyto hodnoty se nazývají \uv{vyhledávací tabulka},
nebo anglicky \uv{lookup table} (\acs{LUT}).
Frekvence výsledného signálu je závislá na prodlevě mezi načítáním jednotlivých hodnot. \cite{ArtOfElectronics}
Signál z převodníku je následně filtrován dolní propustí o adekvátní mezní frekvenci,
která odstraní vyšší harmonické složky vzniklé z diskrétních digitálních kroků.
Princip funkce DDS je ilustrován na obr. \ref{fig:DDS}. 

\begin{figure}[h!]
    \centering
    \includegraphics[width=1\textwidth]{obrazky/DDS_chart.pdf} % Adjusted width for the second figure
    \caption{Ilustrace principu DDS}
    \label{fig:DDS} % Changed label for the second figure
\end{figure}


\subsection{Úprava amplitudy}
Jednoduché integrované obvody DDS typicky nepodporují nastavení amplitudy nebo 
stejnosměrné složky signálu, tuto úpravu je tedy nutno provést externě.

Digitálního řízení amplitudy lze docílit několika způsoby, například:
\begin{itemize}
    \item Zapojit sinusový signál jako referenci \acs{D/A} převodníku, který pak bude fungovat jako digitálně ovládaný dělič. 
    \item Použít zesilovač s programovatelným zesílením (\acs{PGA}). Tyto zesilovače většinou mají málo úrovní zesílení.
    \item Použít operační zesilovač s digitálním potenciometrem ve zpětné vazbě. Takové zapojení je limitováno minimálním zesílením $A=1$.
    \item Operačním zesilovačem zesílit amplitudu na maximální hodnotu a následně ji snížit digitálním potenciometrem zapojeným jako odporový dělič.
    \item Nejdříve zeslabit signál digitálním potenciometrem -- děličem a poté jej zesílit operačním zesilovačem s fixním zesílením. Oproti předchozímu zapojení se tím ušetří jeden sledovač, protože výstup operačního zesilovače je nízkoimpedanční.
\end{itemize}
 

Z těchto metod se jako nejvýhodnější jeví poslední jmenovaná. Její schéma je zobrazeno na obr. \ref{fig:ampl}.

\begin{figure}[h!]
    \centering
    \includegraphics[width=0.7\textwidth]{obrazky/amplitude_set_sch.pdf} % Adjusted width for the second figure
    \caption{Schéma obvodu pro nastavení amplitudy signálu.}
    \label{fig:ampl} % Changed label for the second figure
\end{figure}
Fixní zesílení operačního zesilovače se řídí vztahem \ref{eq:noninv_gain}.
\begin{equation}
    A =\frac{V_{out}}{V_{in}} = 1 + \frac{R_2}{R_3}
    \label{eq:noninv_gain}
\end{equation}
Vstupní amplituda je dána $U_{in} = 0,6 \ \unit{V}$, požadovaná maximální amplituda signálu je $U_{out} =3\ V$.
Dosazením (rovnice \ref{eq:noninv_dosaz}) získáme poměr rezistorů $R_{2}$ a $R_{3}$.
\begin{equation}
    A =\frac{3}{0,6} = 5 = 1 + \frac{4}{1}
    \label{eq:noninv_dosaz}
\end{equation}
Prakticky můžeme použít např. $R_{3} = 10\ k\Omega$ ; $R_{2} = 4\times 10\ k\Omega$ v sérii.

Celková amplituda výstupního signálu $U_{\text{out}}$ vychází z rovnice \ref{eq:outputamp}.
Digitální hodnotu potenciometru pro konkrétní amplitudu $U_{out}$ vypočítáme ze vztahu \ref{eq:digival}.

\begin{equation}
    U_{\text{out}} = U_{\text{in}} \cdot \frac{X}{2^n} \cdot A
\label{eq:outputamp}
\end{equation}
\begin{equation}
        X = \frac{2^n\cdot U_{\text{out}}}{A\cdot U_{\text{in}}} 
    \label{eq:digival}
\end{equation}
\hspace{11em}\text{kde } $n$ \text{ je bitové rozlišení digitálního potenciometru}


\subsection{Úprava stejnosměrné složky signálu}

Integrované obvody DDS typicky generují signál s takovou superponovanou stejnosměrnou složkou, že výstupní napětí je vždy kladné.
Pro základní měření impedance se používá signál bez stejnosměrné složky, je tedy nutno ji odstranit CR filtrem horní propust.

Mezní frekvence filtru je dána vztahem \ref{eq:cutoff_freq}. Hodnoty R a C jsou voleny tak, aby mezní frekvence byla v rozmezí 10 Hz až 30 Hz (rovnice \ref{eq:cutoff_freq_calc}).
\begin{equation}
    f_c = \frac{1}{2\pi R C}
    \label{eq:cutoff_freq}
\end{equation}
\begin{equation}
    f_c = \frac{1}{2\pi \cdot 10000 \cdot 1\cdot 10^{-6}} = 15{,}92\ \unit{Hz}
    \label{eq:cutoff_freq_calc}
\end{equation}

V některých případech je žádoucí k vyfiltrovanému signálu přidat nastavitelnou stejnosměrnou složku. Toho se jednoduše docílí \acs{OZ} v zapojení \uv{neinvertující sčítač}, 
který k signálu přičte nastavitelné napětí $U_{DC}$ z \acs{D/A} převodníku, případně vyfiltrované napětí PWM z mikrokontroléru. Aby filtry nebyly zatíženy, jsou použity \acs{OZ} v zapojení \uv{sledovač}.
Schéma zapojení je zobrazeno na obr. \ref{fig:sch_adder}.

\begin{figure}[h!]
    \centering
    \includegraphics[width=0.85\textwidth]{obrazky/adder_sch.pdf} % Adjusted width for the second figure
    \caption{Schéma obvodu pro úpravu stejnosměrné složky signálu.}
    \label{fig:sch_adder} % Changed label for the second figure
\end{figure}

\section{Měřící Sekce}


\subsection{Výběr operačního zesilovače}
Operační zesilovač v měřící sekci je kritický prvek, odvíjí se od něj přesnost celého měřidla.
Při jeho výběru je nutno dbát na následující parametry:
\begin{enumerate}
    \item \textbf{Šířka pásma (\acs{GBW})} -- U samovyvažovacího můstku musí \acs{OZ} udržovat \uv{virtuální nulu} na svém invertujícím vstupu. 
    Aby byla tato nula dostatečně stabilní i při vyšších frekvencích, musí mít \acs{OZ} velkou rezervu zesílení. 
    \acs{GBW} by tedy mělo být alespoň $100\times$ vyšší, než je maximální měřící frekvence.
    \item \textbf{Vstupní proud} -- Vstupní proud \acs{OZ} by měl být co nejnižší, aby nedocházelo ke zkreslení zejména při měření vysokých impedancí.
    Vstupní proud musí být řádově menší než proud protékající měřenou impedancí, jinak vznikne hrubá chyba.
    \item \textbf{Napěťový offset} -- \acs{OZ} by měl mít co nejmenší napěťový offset, aby neovlivňoval výsledky měření nízkých napětí a nebyla tak do výsledků vnášena systematická chyba.
    \item \textbf{Rychlost přeběhu (slew rate)} -- Rychlost přeběhu musí být dostatečná, aby \acs{OZ} byl schopen sledovat rychle se měnící signál. Aby se zabránilo zkreslení, musí platit:
    \begin{equation}
        SR \geq 2 \cdot \pi \cdot f_{max} \cdot U_p
    \end{equation}
    kde $f_{max}$ je maximální měřená frekvence signálu a $U_p$ je jeho špičková amplituda.
    \item \textbf{Šumové vlastnosti} -- \acs{OZ} by měl mít nízkou úroveň vlastního šumu zejména při měření malých signálů, aby nebyl výsledek měření ovlivněn rušivými komponentami ze zesilovače samotného.
\end{enumerate}


\subsection{Přepínání rozsahu}

Pro dosažení požadované přesnosti měření v širokém rozsahu impedancí je nutné použít více bočníků s různými hodnotami odporu. 
Volba správného bočníku je kritická pro přesnost měření -- pokud je odpor bočníku příliš velký, napětí na něm bude příliš malé a měření bude ovlivněno šumem.
Pokud je odpor příliš malý, může dojít k přetížení operačního zesilovače nebo k nedostatečnému rozlišení při měření vysokých impedancí.

Princip přepínání rozsahu spočívá v tom, že podle očekávané hodnoty měřené impedance se automaticky nebo manuálně zapojí do obvodu vhodný bočník.
Pro měření nízkých impedancí se použije menší bočník (např. $1\ \Omega$, $10\ \Omega$), pro měření vysokých impedancí se použije větší bočník (např. $1\ \text{k}\Omega$, $10\ \text{k}\Omega$).

Pro přepínání bočníků lze použít několik metod:
\begin{itemize}
    \item \textbf{Relé} -- Mechanická relé poskytují nejlepší vlastnosti z hlediska odporu v sepnutém stavu (typicky $< 100\ \text{m}\Omega$) a izolace v rozepnutém stavu.
    Nevýhodou je jejich pomalé přepínání (typicky $5$--$10\ \text{ms}$), omezená životnost a vyšší cena.
    \item \textbf{Analogové spínače (\acs{CMOS})} -- Integrované obvody obsahující více analogových spínačů jsou rychlé, mají dlouhou životnost a nízkou spotřebu.
    Nevýhodou je vyšší odpor v sepnutém stavu (typicky $5$--$100\ \Omega$) a omezená izolace v rozepnutém stavu.
\end{itemize}


Automatické přepínání rozsahu lze implementovat následujícím algoritmem:
\begin{enumerate}
    \item Začít měření s největším bočníkem (pro nejvyšší impedance).
    \item Změřit napětí na bočníku.
    \item Pokud je napětí příliš malé (pod prahovou hodnotou, např. $10\%$ plného rozsahu \acs{A/D} převodníku), přepnout na menší bočník.
    \item Pokud je napětí příliš velké (nad $90\%$ plného rozsahu), přepnout na větší bočník.
    \item Opakovat kroky 2--4, dokud není nalezen vhodný rozsah.
\end{enumerate}

Pro zajištění stability měření je vhodné po přepnutí rozsahu počkat několik period měřícího signálu, než se provede finální měření, aby se přechodové jevy ustálily.

Hodnoty bočníků by měly být voleny tak, aby se jejich rozsahy překrývaly, což umožňuje plynulé přepínání bez mezer v měřitelném rozsahu.
Typicky se používají dekadické hodnoty (např. $1\ \Omega$, $10\ \Omega$, $100\ \Omega$, $1\ \text{k}\Omega$, $10\ \text{k}\Omega$).


