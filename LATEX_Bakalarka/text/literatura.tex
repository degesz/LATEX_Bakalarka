% Pro sazbu seznamu literatury použijte jednu z následujících možností

%%%%%%%%%%%%%%%%%%%%%%%%%%%%%%%%%%%%%%%%%%%%%%%%%%%%%%%%%%%%%%%%%%%%%%%%%
%1) Seznam citací definovaný přímo pomocí prostředí literatura / thebibliography

%\begin{thebibliography}{99}
%	
%\bibitem{sr72/2017}
%	VYSOKÉ UČENÍ TECHNICKÉ V~BRNĚ.
%	\emph{Směrnice č.\,72/2017, Úprava, odevzdávání a~zveřejňování závěrečných prací.}
%	Online. Brno: VUT v~Brně, 2017.
%	Úplné znění ke dni 11.\,4.\,2022.
%	Dostupné z:\\
%	{\small
%	\url{https://www.vut.cz/uredni-deska/vnitrni-predpisy-a-dokumenty/smernice-c-72-2017-uprava-odevzdavani-a-zverejnovani-zaverecnych-praci-d161410}.}
%	[cit.\ 2023-09-27].
%
%\bibitem{CSN_ISO_690-2022}
%    ÚŘAD PRO TECHNICKOU NORMALIZACI, METROLOGII A~STÁTNÍ ZKUŠEBNICTVÍ.
%    ČSN ISO 690:2022 (01 0197), \emph{Informace a dokumentace -- Pravidla pro bibliografické odkazy a~citace informačních zdrojů.}
%    Čtvrté vydání. Praha, 2022.
%
%\bibitem{CSN_ISO_7144-1997}
%    ÚŘAD PRO TECHNICKOU NORMALIZACI, METROLOGII A~STÁTNÍ ZKUŠEBNICTVÍ.
%    ČSN ISO 7144 (010161), \emph{Dokumentace -- Formální úprava disertací a~podobných dokumentů.}
%%    24 stran.
%    Praha, 1997.
%
%\bibitem{CSN_ISO_31-11}
%    ÚŘAD PRO TECHNICKOU NORMALIZACI, METROLOGII A~STÁTNÍ ZKUŠEBNICTVÍ.
%    ČSN ISO 31-11, \emph{Veličiny a~jednotky -- část 11: Matematické znaky a~značky používané ve fyzikálních vědách a~v~technice.}
%    Praha, 1999.
%
%\bibitem{Farkasova23:CSNISO6902022komentar}
%	FARKAŠOVÁ, B.; GARAMSZEGI T.; JANSOVÁ L.; KONEČNÝ L.; KRČÁL M.\ et~al.
%	\emph{Výklad normy ČSN ISO 690:2022 (01 0197) účinné od 1.\,12.\,2022}.
%	 Online. První vydání. 2023.
%	Dostupné~z:
%	\url{https://www.citace.com/Vyklad-CSN-ISO-690-2022.pdf}.
%	[cit.\,2023-09-27].
%
%
%\end{thebibliography}

%%%%%%%%%%%%%%%%%%%%%%%%%%%%%%%%%%%%%%%%%%%%%%%%%%%%%%%%%%%%%%%%%%%%%%%%
%2) Seznam citací pomocí BibTeXu (automatické vytvoření seznamu literatury z databáze)
% V této variantě je třeba spouštět BibTeX zvlášť (není součástí překladu), což ale neplatí pro Overleaf, který BibTex spouští sám.
% Definice ‚stylu‘ seznamu
%\bibliographystyle{unsrturl}
%nebo
\bibliographystyle{czechiso} %pro ISO ČSN 690 (nespoléhejte na něj ale úplně; je to zastaralý soubor a z povahy věci rovněž plyne, že si neporadí se všemi nároky normy; běžnou praxí je nechat Bibtex vygenerovat seznam zdrojů a nakonec jej ručně doupravit do finální podoby.
% Soubor s databází zdrojů
\bibliography{text/literatura}

% Následující příkaz je pouze pro ukázku sazby literatury při použití BibTeXu.
% Způsobí citaci všech zdrojů v souboru literatura.bib, i když nejsou citovány v textu.
%\nocite{*}  %tohle však ve vaší práci nedělejte


%%%%%%%%%%%%%%%%%%%%%%%%%%%%%%%%%%%%%%%%%%%%%%%%%%%%%%%%%%%%%%%%%%%%%%%%%
%%3) Reference generované BibLaTeXem (automatické vytvoření seznamu literatury z databáze)
%Nejmodernější přístup, nicméně doporučujeme pouze zkušenějším uživatelům
%Pro tuto variantu můžete najít styl pro ČSN ISO 690 zde:
%https://github.com/michal-h21/biblatex-iso690
%avšak buďte si vědomi toho, že žádná automatická metoda nemůže vyhovět požadavkům normy.