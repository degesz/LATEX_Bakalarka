\chapter*{Závěr}
\phantomsection
\addcontentsline{toc}{chapter}{Závěr}

Cílem této semestrální práce bylo nastudovat principy měření impedance elektrických součástek a navrhnout koncepci měřiče impedance pro frekvenční pásmo do 100~kHz.

V~teoretické části práce byly popsány základní principy měření impedance a analyzovány různé metody realizace měřicích přístrojů. 
Jako hlavní koncept byl zvolen samo-vyvažovací můstek, který se používá v~moderních komerčních LCR metrech. 
Dále byla zpracována problematika generace měřicího signálu metodou přímé digitální syntézy (DDS) a možnosti úpravy jeho parametrů -- amplitudy a stejnosměrné složky.

Významná část práce byla věnována digitalizaci a zpracování signálů. Byla analyzována jednodušší metoda převodu na stejnosměrná napětí pomocí špičkových a 
fázových detektorů, která se však ukázala jako nevhodná pro přesná měření kvůli citlivosti na šum. 
Proto byla navržena alternativní metoda přímého vzorkování harmonických signálů A/D převodníkem a jejich následného zpracování diskrétní Fourierovou transformací.

Pro ověření navržených konceptů byl zkonstruován první prototyp měřiče. Na tomto prototypu byly úspěšně ověřeny subsystémy napájení a generátoru měřicího signálu. 
Testování však odhalilo několik nedostatků návrhu, což motivovalo přechod na metodu digitalizace přímým vzorkováním.

Na základě získaných poznatků bude práce v~navazující bakalářské práci pokračovat následujícími kroky:
\begin{itemize}
    \item Konstrukce testovacího přípravku pro ověření synchronního vzorkování dvou signálů při vzorkovací frekvenci 1~MHz.
    \item Finální návrh a realizace kompletního měřiče impedance s~vylepšenou analogovou částí a implementací metody přímého vzorkování.
    \item Provedení ověřovacích měření na sadě referenčních součástek a porovnání výsledků s~komerčním laboratorním přístrojem.
    \item Stanovení dosažené přesnosti měření a zhodnocení parametrů zařízení.
\end{itemize}

