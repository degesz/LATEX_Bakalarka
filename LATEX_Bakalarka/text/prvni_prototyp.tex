\chapter{První prototyp}
První pokus o návrh zařízení byl založen na metodě převodu na stejnosměrná napětí, která je blíže popsána v kapitole \ref{sec:digit_DC}.
Tato metoda se však později ukázala jako slepá vývojová větev, protože neposkytovala dostatečnou přesnost měření. 
Pro ověření konceptu byl vytvořen prototyp zařízení (obr. \ref{fig:prototyp_render}).

Prototyp obsahoval pouze měřící část bez uživatelského rozhraní. Naměřená data byla přenášena na počítač pomocí sériové linky.
Původní záměr byl připojit ovládací modul s displejem a tlačítky později jako samostatný modul.




\begin{figure}[htbp]
	\centering
	\begin{subfigure}[b]{0.48\textwidth}
		\centering
		\includegraphics[width=\linewidth]{obrazky/prototyp_render.png}
		\caption{Přední strana}
		\label{fig:prototyp_render_front}
	\end{subfigure}
	\hfill
	\begin{subfigure}[b]{0.48\textwidth}
		\centering
		\includegraphics[width=\linewidth]{obrazky/prototyp_render_back.png}
		\caption{Zadní strana}
		\label{fig:prototyp_render_back}
	\end{subfigure}
	\caption{3D render prvního prototypu}
	\label{fig:prototyp_render}
\end{figure}


\section{Validované subsystémy}
Na prototypu byly odzkoušeny řešení pro napájení a generaci měřícího signálu. Tyto subsystémy fungovaly bezproblémově a byly proto přeneseny do finálního návrhu.

\subsection{Napájecí systém}
Celé zařízení je napájeno z 5V USB. Dále je zařazen jednoduchý lineární regulátor HT7533, který poskytuje napájecí linku 3,3 V pro digitální součástky.

Jelikož analogová část obvodu pracuje se střídavým napětím, potřebují \acs{OZ} kladnou i zápornou napájecí linku.
Tu zajišťuje integrovaný obvod LM27762 od společnosti Texas Instruments \cite{LM27762}.
Tento čip v sobě sdružuje invertující nábojovou pumpu a lineární regulátory pro obě koleje.
Schéma napájecího systému je zobrazeno na obr.~\ref{fig:power_sch}.

\begin{figure}[h!]
    \centering
    \includegraphics[width=1\textwidth]{obrazky/power_sch.pdf}
    \caption{Schéma napájecího systému prvního prototypu.}
    \label{fig:power_sch}
\end{figure}

Rezistory R4 až R7 tvoří děliče zpětné vazby a nastavují výstupní napětí regulátorů.

Výstupní napětí kladné koleje je dáno vztahem \ref{eq:vout_pos}.
\begin{equation}
    V_{OUT+} = 1{,}2\ \unit{V} \cdot \frac{R_4 + R_5}{R_5}
    \label{eq:vout_pos}
\end{equation}
Analogicky pro zápornou kolej platí vztah \ref{eq:vout_neg}.
\begin{equation}
    V_{OUT-} = -1{,}2\ \unit{V} \cdot \frac{R_6 + R_7}{R_7}
    \label{eq:vout_neg}
\end{equation}

Zvolením hodnot $R_4 = R_6 = 68\ \text{k}\Omega$ a $R_5 = R_7 = 22\ \text{k}\Omega$ je regulace výstupu nastavena na $\pm 4{,}91\ \unit{V}$.
Tím je zajištěn dostatečný prostor pro úbytek napětí na regulátorech. \cite{LM27762}


\subsection{Generátor měřícího signálu}




\section{Zjištěné nedostatky}



\subsection{}
