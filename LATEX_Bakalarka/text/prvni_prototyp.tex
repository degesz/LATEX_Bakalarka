\chapter{První prototyp}

Pro ověření proveditelnosti a funkčnosti konceptu byl vytvořen první prototyp zařízení.
Prototyp obsahoval pouze měřící část bez uživatelského rozhraní. Naměřená data byla přenášena na počítač pomocí sériové linky.



Na prototypu byly také odzkoušeny řešení pro napájení a generaci měřícího signálu. Tyto subsystémy fungovaly bezproblémově, byly tedy přeneseny do finálního designu.



\begin{figure}[htbp]
	\centering
	\begin{subfigure}[b]{0.48\textwidth}
		\centering
		\includegraphics[width=\linewidth]{obrazky/prototyp_render.png}
		\caption{Přední strana}
		\label{fig:prototyp_render_front}
	\end{subfigure}
	\hfill
	\begin{subfigure}[b]{0.48\textwidth}
		\centering
		\includegraphics[width=\linewidth]{obrazky/prototyp_render_back.png}
		\caption{Zadní strana}
		\label{fig:prototyp_render_back}
	\end{subfigure}
	\caption{3D render prvního prototypu}
	\label{fig:prototyp_render}
\end{figure}


\section{Zjištěné nedostatky}
K digitalizaci signálu byla použita metoda převodu na stejnosměrná napětí, která je blíže popsána v kapitole \ref{sec:digit_DC}.
Tento přístup byl zvolen zpočátku intuitivně, avšak výsledky zkoušení prototypu odhalily jeho nedostatky. To vedlo k výzkumu dalších metod digitalizace.

\subsection{}
